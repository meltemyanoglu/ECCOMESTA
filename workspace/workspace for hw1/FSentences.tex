\section{Filtered Sentences}
%Organize, in a schematic way, the information that can be extracted from the Natural Language Sentences

A company that owns a chain of supermarkets wants to acquire a new database system in order to store data about their supermarkets, products, and clients.

\vspace{5mm}

% NEED TO REMOVE THE SANGRIA HERE --------------------------------------------------------------------------------------------------------------------------------------
\noindent
Each supermarket will store the next information:


\begin{itemize}
    \item \textbf{Shop code}: to identify it from other establishments
    \item \textbf{Location}: which will include region, city and address
    \item \textbf{Employees}: a list with every current employee as well as personal data about each one
    \item \textbf{Sales record}: with recent information about sales
\end{itemize}

\vspace{5mm}
\noindent
The users that will have access to the database are:


\begin{itemize}
    \item \textbf{Stock clerks}: updating the database by removing expired products and taking stock every day
    \item \textbf{Store Manager:} editing the database as needed, can see all of the employee and the products in his shop
    \item \textbf{Inventory Control Specialists: }in charge of adding the new products when the merchandise arrives and keeping the stock level in order by placing orders of products in need
    \item \textbf{Data Analysts:} view the database in order to make statistics and to study the performance of the supermarkets
    \item \textbf{Human Resources Personnel: }adding new employees when they are hired, promoting them and removing them from the database when they are fired
    \item \textbf{IT Specialists: }administrate the database, keeping it in order and functional
\end{itemize}
Each role contributes to the seamless operation of the supermarket chain, from customer transactions to product management and data analysis.

\vspace{5mm}
\noindent
Products are characterized by:

\begin{itemize}
    \item \textbf{Product code}: to identify it from other products
    \item \textbf{Name}: a brief description of the product. It can be printed in the receipt
    \item \textbf{Price}: shared across all the supermarkets
    \item \textbf{VAT}: percentage of tax (Standard rate: 22\% Reduced rates: 10\% and 5\% Super-reduced rate: 4\%)
    \item \textbf{Category}: to identify the type of product (frozen, cosmetic, alcohol, etc.)
    \item \textbf{Supplier code}: to identify the supplier in order to reorder the product
\end{itemize}

\vspace{5mm}
\noindent
Furthermore, each supermarket location will store the next information related to products:

\begin{itemize}
    \item \textbf{Stock level}: quantity of an specific product in the store
    \item \textbf{Re-Order level:} minimum quantity that must be available in a supermarket
    \item \textbf{Order Quantity}: quantity that is ordered from the supplier when stock level is low
\end{itemize}

\vspace{5mm}
\noindent
Receipts will also be stored during one month, including:

\begin{itemize}
    \item \textbf{List of products:} with the quantity of each one
    \item \textbf{Date}: year, month, day and hour
    \item \textbf{Pay method: }cash or card (with the needed information about each one)
    \item \textbf{Supermarket location}
\end{itemize}

\vspace{5mm}
\noindent
The data stored about clients registered in the loyalty program is:

\begin{itemize}
    \item \textbf{Email}: to identify each user
    \item \textbf{Password}: used by the user to access its account
    \item \textbf{\textbf{Name and surname}}
    \item \textbf{\textbf{Date of birth}}
    \item \textbf{Points}: earned by shopping. Can be exchanged for discounts
    \item \textbf{Prefered supermarket location}: to inform the user about that supermarket’s discounts
    \item \textbf{Receipts}: made using loyalty program
\end{itemize}
 