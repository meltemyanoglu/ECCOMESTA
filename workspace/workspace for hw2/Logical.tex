\section{Logical Design}

\subsection{Transformation of the Entity-Relationship Schema}


\subsubsection{Redundancy Analysis }

\textbf{Intentional redundancy}: storing the same data in different formats or representations. 

\textbf{Extensional redundancy}: storing the same data in more than one place without a specific purpose or benefit.



\subsubsection{Choice of Principal Identifiers}

There are no circular dependencies among the weak entities and their owners in the schema, and the primary keys follow the best practices for choosing identifiers.

\subsection{Analysis of Database Load}
Since there is not any derived attribute in the provided ER-Schema, we provide the load analysis as if the Customer entity had the Customer Lifetime Value as derived attribute. Consider the following two operations that involve the redundant attribute CLValue:
\begin{itemize}
    \item O1: Store the transaction data into the loyalty program;

    \begin{table}
        \centering
        \begin{tabular}{cccccc}
             &  &  &  &  & \\
             &  &  &  &  & \\
             &  &  &  &  & \\
             &  &  &  &  & \\

        \begin{table}
            \centering
            \begin{tabular}{ccccc}

                 \begin{table}
                     \centering
                     \begin{tabular}{c}
                         \\
                     \end{tabular}
                     \caption{Caption}
                     \label{tab:my_label}
                 \end{table}
                                  &  &  &  & \\
                 &  &  &  & \\
                 &  &  &  & \\
                 &  &  &  & \\
            \end{tabular}
            \caption{Caption}
            \label{tab:my_label}
        \end{table}
                \end{tabular}
        \caption{Caption}
        \label{tab:my_label}
    \end{table}
        \item O2: Calculate the CLV data for a specific customer in the loyalty program; 
\end{itemize}

\begin{longtable}{|p{.22\columnwidth}|p{.43\columnwidth} |p{.13\columnwidth}|p{.12\columnwidth} |}

\hline
\textbf{Operation} & \textbf{Description} & \textbf{Frequency} & \textbf{Type}  \\\hline

O1: Store the transaction data &  Store the transaction data of a customer in the loyalty program if he/she enrolled & 20000/day & Online \\\hline

O2: Calculate the CLV & Calculate the Lifetime value of the specific customer & 1/month & Offline \\\hline

\caption{Frequency Table}
\end{longtable}


\begin{longtable}{|p{.20\columnwidth}|p{.20\columnwidth} |p{.16\columnwidth}|p{.16\columnwidth}|p{.16\columnwidth} |}

\hline
\textbf{Concept} & \textbf{Construct} & \textbf{Access} & \textbf{Type} & \textbf{Average Access}  \\\hline

Lorem ipsum &  Lorem ipsum & Lorem ipsum & Lorem ipsum & Lorem ipsum \\\hline

Lorem ipsum &  Lorem ipsum & Lorem ipsum & Lorem ipsum & Lorem ipsum \\\hline

\caption{Frequency Table}
\end{longtable}


\begin{longtable}{|p{.20\columnwidth}|p{.20\columnwidth} |p{.16\columnwidth}|p{.16\columnwidth}|p{.16\columnwidth} |}

\hline
\textbf{Concept} & \textbf{Construct} & \textbf{Access} & \textbf{Type} & \textbf{Average Access}  \\\hline

Lorem ipsum &  Lorem ipsum & Lorem ipsum & Lorem ipsum & Lorem ipsum \\\hline

Lorem ipsum &  Lorem ipsum & Lorem ipsum & Lorem ipsum & Lorem ipsum \\\hline

\caption{Frequency Table}
\end{longtable}


\begin{longtable}{|p{.20\columnwidth}|p{.20\columnwidth} |p{.16\columnwidth}|p{.16\columnwidth}|p{.16\columnwidth} |}

\hline
\textbf{Concept} & \textbf{Construct} & \textbf{Access} & \textbf{Type} & \textbf{Average Access}  \\\hline

Lorem ipsum &  Lorem ipsum & Lorem ipsum & Lorem ipsum & Lorem ipsum \\\hline

Lorem ipsum &  Lorem ipsum & Lorem ipsum & Lorem ipsum & Lorem ipsum \\\hline

\caption{Frequency Table}
\end{longtable}


\begin{longtable}{|p{.20\columnwidth}|p{.20\columnwidth} |p{.16\columnwidth}|p{.16\columnwidth}|p{.16\columnwidth} |}

\hline
\textbf{Concept} & \textbf{Construct} & \textbf{Access} & \textbf{Type} & \textbf{Average Access}  \\\hline

Lorem ipsum &  Lorem ipsum & Lorem ipsum & Lorem ipsum & Lorem ipsum \\\hline

Lorem ipsum &  Lorem ipsum & Lorem ipsum & Lorem ipsum & Lorem ipsum \\\hline

\caption{Frequency Table}
\end{longtable}s


\begin{longtable}{|p{.20\columnwidth}|p{.20\columnwidth} |p{.16\columnwidth}|p{.16\columnwidth}|p{.16\columnwidth} |}

\hline
\textbf{Concept} & \textbf{Construct} & \textbf{Access} & \textbf{Type} & \textbf{Average Access}  \\\hline

Lorem ipsum &  Lorem ipsum & Lorem ipsum & Lorem ipsum & Lorem ipsum \\\hline

Lorem ipsum &  Lorem ipsum & Lorem ipsum & Lorem ipsum & Lorem ipsum \\\hline

\caption{Frequency Table}
\end{longtable}


\subsection{Relational Schema}
\includegraphics[scale=0.45, angle=90]{pictures/FDB - HW2 - Relational Schema (1).jpeg}

\subsection{Data Dictionary}

\begin{longtable}{|p{.15\columnwidth}|p{.15\columnwidth} |p{.30\columnwidth}|p{.10\columnwidth}|p{.20\columnwidth} |} 
\hline
\textbf{Relation} & \textbf{Attribute} & \textbf{Description} & \textbf{Domain} & \textbf{Constraints} \\\hline


\multirow{5}{*}{\textbf{STORES}} & StockQuantity & Quantity available in the store & Int  & Not NULL  \\\cline{2-5}
& ReOrderLevel & Minimum quantity that must be in the store & Int & Not NULL \\\cline{2-5} 
& OrderQuantity  & Quantity of product that will be ordered & Int & Not NULL \\\cline{2-5}
& ProductID & Identification of the product & Serial & Foreign key to Product, not NULL, primary key with StoreID \\\cline{2-5}
& StoreID & Identification of the store & Serial & Foreign key to Store, not NULL, primary key with ProductID \\\hline

\multirow{7}{*}{\textbf{PRODUCT}} 
&ProductID &Identification of the product & Serial&Primary Key \\\cline{2-5}
&UPC &Universal Product Code, standardized barcode symbology used for tracking items & Int ??? & Not NULL\\\cline{2-5}
&Name &Name of the product&Text&Not NULL \\\cline{2-5}
&VAT &Value Added Tax &Double ??? &Not NULL \\\cline{2-5}
&Price &Price of the product &Int &Not NULL \\\cline{2-5}
&Category &Category of the product & Text ??& Not NULL\\\cline{2-5}
&SupplierID &Identification of the supplier of the product &Serial & Not NULL\\\hline

\multirow{1}{*}{\textbf{CATEGORY}} 
&CategoryName &Name of the category & Text ??& Primary Key, Not NULL\\\hline

\multirow{6}{*}{\textbf{RECEIPT}} 
&ReceiptID &Identification of the receipt &Serial? & Primary Key, Not NULL\\\cline{2-5}
&TotalAmount &Total price of the purchase &Int & Not NULL\\\cline{2-5}
&Date &Date of the purchase &Datetime* &Not NULL \\\cline{2-5}
&PaymentMethod &Type of payment used &Text ?? &Not NULL \\\cline{2-5}
&CustomerID &Identification of the customer & Serial&CAN BE NULL IF THE PERSON ID NOT REGISTERED?? \\\cline{2-5}
&StoreID &Identification of the store &Serial &Not NULL \\\hline

\multirow{4}{*}{\textbf{PROMOTION}} 
&Start-EndDates &Initial and final date of a promotion &Date &Not NULL \\\cline{2-5}
&DiscountInfo &Information or description of the promotion  &Text ?? &Not NULL \\\cline{2-5}
&StoreID &Identification of the store that has the promotion &Serial & Foreign key to Store, not NULL, primary key with ProductID\\\cline{2-5}
&ProductID &Identification of the product promoted &Serial &Foreign key to Product, not NULL, primary key with StoreID \\\hline


\multirow{8}{*}{\textbf{CUSTOMER}} 
&Email &Email of the customer &Text &Primary Key, Not NULL \\\cline{2-5}
&Name &Name of the customer &Text &Not NULL \\\cline{2-5}
&Surname &Surname of the customer &Text &Not NULL \\\cline{2-5}
&Gender &Gender of the customer &Text &Not NULL \\\cline{2-5}
&Password &Password of the customer &Text &Not NULL \\\cline{2-5}
&DateOfBirth &Date of Birth of the customer &Date &Not NULL \\\cline{2-5}
&LoyalityPoints &Number of loyalty points acquired &Int &Not NULL \\\cline{2-5}
&StoreID &Identification of the preference store &Serial &Not NULL \\\hline

\multirow{3}{*}{\textbf{CONTAINS}} 
&Quantity &Quantity of product bought &Int ?? what about grams &Not NULL \\\cline{2-5}
&ReceiptID &Identification of the receipt &Serial &Foreign key to Receipt, not NULL, primary key with ProductID  \\\cline{2-5}
&ProductID &Identification of the product &Serial &Foreign key to Product, not NULL, primary key with ReceiptID \\\hline

\multirow{4}{*}{\textbf{SUPPLIER}} 
&SupplierID &Identification of the supplier &Serial &Primary Key, Not NULL \\\cline{2-5}
&Address &Address of the supplier & &Not NULL \\\cline{2-5}
&ContactInfo &Contact Information of the supplier & &Not NULL \\\cline{2-5}
&ContractDetails &Details and description of the contract between store and supplier &Text &Not NULL \\\hline

\multirow{2}{*}{\textbf{SUPPLIES TO}} 
&SupplierID &Identification of the supplier&Serial &Foreign key to Supplier, not NULL, primary key with StoreID \\\cline{2-5}
&StoreID &Identification of the store &Serial &Foreign key to Store, not NULL, primary key with SupplierID \\\hline

\multirow{4}{*}{\textbf{STORE}} 
&StoreID &Identification of the store &Serial &Primary key, Not NULL \\\cline{2-5}
&Location &Address of the store & Text&Not NULL \\\cline{2-5}
&Schedule &Schedule &Timetable of the store &Not NULL \\\cline{2-5}
&ManagerID &Identification of the manager of the store &Serial &Not NULL \\\hline

\multirow{1}{*}{\textbf{POSITION}} 
&PositionName &Name of the position held by the employees &Text ?? & Primary Key, Not NULL \\\hline

\multirow{7}{*}{\textbf{EMPLOYEE}} 
&EmployeeID &Identification of the worker &Serial &Primary Key, Not NULL \\\cline{2-5}
&Name &Name of the employee &Text &Not NULL \\\cline{2-5}
&Surname &Surname of the employee &Text &Not NULL \\\cline{2-5}
&HireDate &Date of the hire of the employee &Date or Datetime ? &Not NULL \\\cline{2-5}
&Salary &Salary of the employee &Double ?? &Not NULL \\\cline{2-5}
&Position &Name of their work position &Text? &Not NULL \\\cline{2-5}
&WorkStore &Identification of the store they work in &Serial &Not NULL \\\hline


\end{longtable}



\subsection{External Constraints}
https://www.overleaf.com/project/65494d6e2dfdbe59123e4dc4


What happens if a customer doesnt have an email or if doesn't wanna be part of the loyality programm ???????


\begin{itemize}
    \item Only the employees with the position of ‘Manager’ can check the information about other employees in the shop. They need to login in the system with their Manager identificator.
    \item HR employees can edit the info of the employees.
    \item Authorized employee can check and edit the information of the products.
    \item Backups are meant to be done every shift change in order to avoid mistakes.
    \item The employees of a store can see the products' quantities of another store but cannot change them.
    \item Only one person can enter stock into a store. (login information is requested.)?? Can anyone view stocks? Do I need to log in to make changes?
    \item No one except the stock tracker can see the stocks in the store. It cannot be renewed or changed.
    \item Only 1 promotion can be applied to a product at the same store. Two promotions cannot be applied at the same time. This means that we cannot apply other promotions to the product until the previous one reaches its ending date. (Really like this one)
\end{itemize}

 