\section{Logical Design}

\subsection{Transformation of the Entity-Relationship Schema}


\subsubsection{Redundancy Analysis}

\textbf{Intentional redundancy}: The schema does not contain any intentional redundancy. \\
\textbf{Extensional redundancy}: The schema does not contain any cycle of entities.

\subsubsection{Choice of Principal Identifiers}

There are no circular dependencies among the weak entities and their owners in the schema, and the primary keys follow the best practices for choosing identifiers.

\subsection{Analysis of Database Load}
Assuming that the Customer entity had a derived attribute for the Customer Lifetime Value, we perform the load analysis on the given ER-Schema. We consider two operations that involve the redundant attribute CLV: 

\begin{itemize}
    \item O1: store the transaction data in the loyalty program;
    \item O2: Calculate the CLV data for a specific customer in the loyalty program; 
\end{itemize}

\begin{center}
\includegraphics[scale=0.23]{pictures/Customer_CLV.png}
\end{center}

Table 4 reports the description of each operation, its frequency and type. O1 is an online operation since the customer's receipt needs to be stored right after the transaction while O2 is related to statistical analysis conducted on the data and is batch operations.


\begin{longtable}{|p{.22\columnwidth}|p{.43\columnwidth} |p{.13\columnwidth}|p{.12\columnwidth} |}

\hline
\textbf{Operation} & \textbf{Description} & \textbf{Frequency} & \textbf{Type}  \\\hline

O1: Store the transaction data &  Store the transaction data of customers in the loyalty program if he/she enrolled & 10000/month & Online \\\hline

O2: Calculate the CLV & Calculate the Lifetime value of a specific customer & 10/month & Batch \\\hline

\caption{Frequency Table}
\end{longtable}


Table 5 shows how often and in what way O1 accesses the data with redundancy. The Customer entity reads the data to determine the current number of receipts or current total amount of how much the customer spends and writes the data to change this number.

\newpage
\begin{longtable}{|p{.20\columnwidth}|p{.20\columnwidth} |p{.14\columnwidth}|p{.14\columnwidth}|p{.20\columnwidth} |}

\hline
\multicolumn{5}{|c|}{\textbf{Operation O1: 10000/month}}\\\hline

\textbf{Concept} & \textbf{Construct} & \textbf{Access} & \textbf{Type} & \textbf{Average Access}\\\hline
Receipt & Entity & 1 & W & 1x10000x2 = 20000 \\\hline
Belongs\_To & Relationship & 1 & W & 1x10000x2 = 20000 \\\hline
Customer & Entity & 1 & R & 1x10000x1 = 10000 \\\hline
Customer & Entity & 1 & W & 1x10000x2 = 20000 \\\hline

\multicolumn{4}{|c|}{\textbf{Total Access}} & \textbf{70000} \\\hline

\caption{Access/volume Table for O1 with redundancy}
\end{longtable}

Table 6 shows how O2 uses the data that is with redundancy. Because of the redundancy, O2 only needs to access the Customer entity once to get all the information it needs.


\begin{longtable}{|p{.20\columnwidth}|p{.20\columnwidth} |p{.14\columnwidth}|p{.14\columnwidth}|p{.20\columnwidth} |}

\hline
\multicolumn{5}{|c|} {\textbf{Operation O2: 10/month}}\\\hline

\textbf{Concept} & \textbf{Construct} & \textbf{Access} & \textbf{Type} & \textbf{Average Access}  \\\hline

Customers &  Entity & 1 & R & 1x10x1 = 10  \\\hline

\multicolumn{4}{|c|}{\textbf{Total Access}}  & \textbf{10} \\\hline


\caption{Access/volume Table for O2 with redundancy}
\end{longtable}

In Table 7 we report the access/volume data related to O1 without redundancy. In this case we have to consider the insertion of a new instance in Customer, and the insertion of a new instance in Belongs\_To to store the receipt the customer had.

\begin{longtable}{|p{.20\columnwidth}|p{.20\columnwidth} |p{.14\columnwidth}|p{.14\columnwidth}|p{.20\columnwidth} |}

\hline

\multicolumn{5}{|c|}{\textbf{Operation O1: 10000/month}} \\\hline
\textbf{Concept} & \textbf{Construct} & \textbf{Access} & \textbf{Type} & \textbf{Average Access}  \\\hline
Receipt & Entity  & 1 & W & 1x10000x2 = 20000 \\\hline
Belongs\_To & Relationship & 1 & W & 1x10000x2 = 20000 \\\hline
\multicolumn{4}{|c|}{\textbf{Total Access}}  & \textbf{40000}  \\\hline

\caption{Access/volume Table for O1 without redundancy}
\end{longtable}


Table 8 shows how O2 uses the data that is not repeated. We assumed that each customer has an average of 10 receipts per month. 

\begin{longtable}{|p{.20\columnwidth}|p{.20\columnwidth} |p{.14\columnwidth}|p{.14\columnwidth}|p{.20\columnwidth} |}

\hline

\multicolumn{5}{|c|} {\textbf{Operation O2: 10/month}} \\\hline
\textbf{Concept} & \textbf{Construct} & \textbf{Access} & \textbf{Type} & \textbf{Average Access}  \\\hline
Customers & Entity & 1 & R & 1x10x1 = 10 \\\hline
Belongs\_To & Relationship & 10 & R & 10x10x1 = 100 \\\hline
\multicolumn{4}{|c|}{\textbf{Total Access}}  & \textbf{110}  \\\hline

\caption{Access/volume Table for O2 without redundancy}
\end{longtable}

In Table 9, we report the final access counts with and without redundancy. According to the obtained results, we can clearly see that removing the redundant attribute from the Customer entity improves the database load. 

\begin{longtable}{|p{.20\columnwidth}|p{.20\columnwidth} |p{.24\columnwidth} |}

\hline

\multicolumn{3}{|c|} {\textbf{Comparison}}  \\\hline
\textbf {Operation} & \textbf{With Redundancy}  &\textbf{Without Redundancy}  \\\hline
O1 & 70000 & 40000 \\\hline
O2 & 10 & 110 \\\hline
\textbf{Total access} & \textbf{70010} & \textbf{40110}  \\\hline

\caption{Comparison of the number of accesses for each operation}
\end{longtable}

\subsection{Relational Schema}
\begin{center}
\includegraphics[scale=0.11, angle=90]{pictures/FDB - HW2 - Relational Schema (3) .png}
\end{center}

Not all of our customers are going to be part of our loyalty program, so to avoid systematic NULL values, we have decided that Belongs-To relationship should have its own relation.

\subsection{Data Dictionary}

\begin{longtable}{|p{.15\columnwidth}|p{.15\columnwidth} |p{.30\columnwidth}|p{.10\columnwidth}|p{.20\columnwidth} |} 
\hline
\textbf{Relation} & \textbf{Attribute} & \textbf{Description} & \textbf{Domain} & \textbf{Constraints} \\\hline


\multirow{5}{*}{\textbf{STORES}} & StockQuantity & Quantity available in the store & Int  & Not NULL  \\\cline{2-5}
& ReOrderLevel & Minimum quantity that must be in the store & Int & Not NULL \\\cline{2-5} 
& OrderQuantity  & Quantity of product that will be ordered & Int & Not NULL \\\cline{2-5}
& ProductID & Identification of the product & Serial & Foreign key to Product, not NULL, primary key with StoreID \\\cline{2-5}
& StoreID & Identification of the store & Serial & Foreign key to Store, not NULL, primary key with ProductID \\\hline

\multirow{7}{*}{\textbf{PRODUCT}} 
&ProductID &Identification of the product & Serial&Primary Key \\\cline{2-5}
&UPC &Universal Product Code, standardized barcode symbology used for tracking items & Int ??? & Not NULL\\\cline{2-5}
&Name &Name of the product&Text&Not NULL \\\cline{2-5}
&VAT &Value Added Tax &Double ??? &Not NULL \\\cline{2-5}
&Price &Price of the product &Int &Not NULL \\\cline{2-5}
&Category &Category of the product & Text ??& Not NULL\\\cline{2-5}
&SupplierID &Identification of the supplier of the product &Serial & Not NULL\\\hline

\multirow{1}{*}{\textbf{CATEGORY}} 
&CategoryName &Name of the category & Text ??& Primary Key, Not NULL\\\hline

\multirow{6}{*}{\textbf{RECEIPT}} 
&ReceiptID &Identification of the receipt &Serial? & Primary Key, Not NULL\\\cline{2-5}
&TotalAmount &Total price of the purchase &Int & Not NULL\\\cline{2-5}
&Date &Date of the purchase &Datetime* &Not NULL \\\cline{2-5}
&PaymentMethod &Type of payment used &Text ?? &Not NULL \\\cline{2-5}
&CustomerID &Identification of the customer & Serial&CAN BE NULL IF THE PERSON ID NOT REGISTERED?? \\\cline{2-5}
&StoreID &Identification of the store &Serial &Not NULL \\\hline

\multirow{4}{*}{\textbf{PROMOTION}} 
&Start-EndDates &Initial and final date of a promotion &Date &Not NULL \\\cline{2-5}
&DiscountInfo &Information or description of the promotion  &Text ?? &Not NULL \\\cline{2-5}
&StoreID &Identification of the store that has the promotion &Serial & Foreign key to Store, not NULL, primary key with ProductID\\\cline{2-5}
&ProductID &Identification of the product promoted &Serial &Foreign key to Product, not NULL, primary key with StoreID \\\hline


\multirow{8}{*}{\textbf{CUSTOMER}} 
&Email &Email of the customer &Text &Primary Key, Not NULL \\\cline{2-5}
&Name &Name of the customer &Text &Not NULL \\\cline{2-5}
&Surname &Surname of the customer &Text &Not NULL \\\cline{2-5}
&Gender &Gender of the customer &Text &Not NULL \\\cline{2-5}
&Password &Password of the customer &Text &Not NULL \\\cline{2-5}
&DateOfBirth &Date of Birth of the customer &Date &Not NULL \\\cline{2-5}
&LoyalityPoints &Number of loyalty points acquired &Int &Not NULL \\\cline{2-5}
&StoreID &Identification of the preference store &Serial &Not NULL \\\hline

\multirow{3}{*}{\textbf{CONTAINS}} 
&Quantity &Quantity of product bought &Int ?? what about grams &Not NULL \\\cline{2-5}
&ReceiptID &Identification of the receipt &Serial &Foreign key to Receipt, not NULL, primary key with ProductID  \\\cline{2-5}
&ProductID &Identification of the product &Serial &Foreign key to Product, not NULL, primary key with ReceiptID \\\hline

\multirow{4}{*}{\textbf{SUPPLIER}} 
&SupplierID &Identification of the supplier &Serial &Primary Key, Not NULL \\\cline{2-5}
&Address &Address of the supplier & &Not NULL \\\cline{2-5}
&ContactInfo &Contact Information of the supplier & &Not NULL \\\cline{2-5}
&ContractDetails &Details and description of the contract between store and supplier &Text &Not NULL \\\hline

\multirow{2}{*}{\textbf{SUPPLIES TO}} 
&SupplierID &Identification of the supplier&Serial &Foreign key to Supplier, not NULL, primary key with StoreID \\\cline{2-5}
&StoreID &Identification of the store &Serial &Foreign key to Store, not NULL, primary key with SupplierID \\\hline

\multirow{4}{*}{\textbf{STORE}} 
&StoreID &Identification of the store &Serial &Primary key, Not NULL \\\cline{2-5}
&Location &Address of the store & Text&Not NULL \\\cline{2-5}
&Schedule &Schedule &Timetable of the store &Not NULL \\\cline{2-5}
&ManagerID &Identification of the manager of the store &Serial &Not NULL \\\hline

\multirow{1}{*}{\textbf{POSITION}} 
&PositionName &Name of the position held by the employees &Text ?? & Primary Key, Not NULL \\\hline

\multirow{7}{*}{\textbf{EMPLOYEE}} 
&EmployeeID &Identification of the worker &Serial &Primary Key, Not NULL \\\cline{2-5}
&Name &Name of the employee &Text &Not NULL \\\cline{2-5}
&Surname &Surname of the employee &Text &Not NULL \\\cline{2-5}
&HireDate &Date of the hire of the employee &Date or Datetime ? &Not NULL \\\cline{2-5}
&Salary &Salary of the employee &Double ?? &Not NULL \\\cline{2-5}
&Position &Name of their work position &Text? &Not NULL \\\cline{2-5}
&WorkStore &Identification of the store they work in &Serial &Not NULL \\\hline


\end{longtable}



\subsection{External Constraints}

\begin{itemize}
    \item Only the employees with the position of ‘Manager’ can check the information about other employees in the shop. They need to login in the system with their Manager identification credentials.
    \item Human Resources employees can edit the information of the employees, as well as add or delete them.
    \item Authorized employees can check and edit the information of the products.
    \item Backups are meant to be done every shift change in order to avoid mistakes.
    \item The employees of a store can see the products' quantities of another store but cannot change them.
    \item If authorized employees of a store wants to view or update stock inventory into a store, login information is requested.
    \item Only 1 promotion can be applied to a product at the same time in a store. Two promotions cannot be applied at the same time. This means that we cannot apply other promotions to the product until the previous one reaches its ending date.
\end{itemize}

 