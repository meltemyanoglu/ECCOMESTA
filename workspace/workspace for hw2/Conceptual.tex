\section{Conceptual Design}

\subsection{Variations to the Requirement Analysis}


Several modifications are made to the functional requirements and explained in details. It also divided into semantically related headings for better understanding. The new list of functional requirements is included in the final section regarding the functional requirements satisfaction check.

\subsection{Entity-Relationship Schema}

\begin{center}
\includegraphics[scale=0.28]{pictures/FDB-HW2-ERDiagram_vers04.png}
\end{center}

\subsection{Data Dictionary}

\subsubsection{Entities Table}
\begin{longtable}{|p{.14\columnwidth}|p{.20\columnwidth} |p{.45\columnwidth}|p{.15\columnwidth} |} 
\hline
\textbf{Entity} & \textbf{Description} & \textbf{Attributes} & \textbf{Identifier}  \\\hline


Product & Items sold in the supermarkets & \begin{itemize}
        \vspace{-1.5em}
        \item ProductID: unique identifier (serial)
        \item Name: brief description of the product (text)
        \item UPC: Universal Product Code (serial)
        \item VAT: Value-Added Tax (float)
        \item Price: the amount customer has to pay (float)
    \end{itemize}
 &  ProductID \\\hline
 
 Store & Where customers buy products & \begin{itemize}
        \vspace{-1.5em}
        \item StoreID: unique identifier (serial)
        \item Location: the place where store is (text)
        \item Schedule: the store timetable (datetime)
    \end{itemize}
 &  StoreID \\\hline

  Receipt & Contains information about each purchase & \begin{itemize}
        \vspace{-1.5em}
        \item ReceiptID: unique identifier (serial)
        \item Date: the time when the payment is done (datetime)
        \item TotalPrice: the amount customer paid for all products (float)
        \item PaymentMethod: a way of how a customer made the payment (text)
    \end{itemize}
 &  ReceiptID \\\hline

  Supplier & A company that supplies the product & \begin{itemize}
        \vspace{-1.5em}
        \item SupplierID: unique identifier (serial)
        \item Address: the place where supplier is (text)
        \item Phone: the telephone number of the supplier (text)
        \item E-mail: Electronic mail address of the supplier (text)
        \item ContractInfo: details of the contract between supplier and store (text)
    \end{itemize}
 &  SupplierID \\\hline

  Customer & A client registered in the loyalty program & \begin{itemize}
        \vspace{-1.5em}
        \item E-mail: unique identifier (text)
        \item Name: name of the customer (text)
        \item Surname: surmane of the customer (text)
        \item DateOfBirth: birth date of the customer (datetime)
        \item Gender: Gender of the customer (text)
        \item Password: the code required to enter the loyalty program (text)
        \item LoyaltyPoints: represent the points of customer registered in the loyalty program (integer)
    \end{itemize}
 &  E-mail \\\hline

  Category & Categories to which products belong & \begin{itemize}
        \vspace{-1.5em}
        \item CategoryName: unique identifier (text)
    \end{itemize}
 &  CategoryName \\\hline 

  Employee & A person who works in a store & \begin{itemize}
        \vspace{-1.5em}
        \item EmployeeID: unique identifier (serial)
        \item Name (text)
        \item Surname (text)
        \item HireDate: a time when an employee hired (datetime)
        \item Salary: amount to be paid an employee monthly (float)
    \end{itemize}
 &  EmployeeID \\\hline

  Position & A title of an employee & \begin{itemize}
        \vspace{-1.5em}
        \item PositionName: unique identifier (text)
    \end{itemize}
 &  PositionName \\\hline



\end{longtable}


\subsubsection{Relationships Table}
\begin{longtable}{|p{.15\columnwidth}|p{.35\columnwidth} |p{.27\columnwidth}|p{.18\columnwidth} |} 
\hline
\textbf{Relationship} & \textbf{Description} & \textbf{Component Entities} & \textbf{Attributes} \\\hline


 Stores & Number of a specific product stored in a store, its re-order level and the order quantity & \begin{itemize}
        \vspace{-1.5em}
        \item Product (0,N)
        \item Store (0,N)
    \end{itemize}
 &  StockQuantity
 
 ReOrderLevel
 
 OrderQuantity\\\hline
 
 Contains & Products bought that are printed in the receipt & \begin{itemize}
        \vspace{-1.5em}
        \item Product (0,N)
        \item Receipt (1,N)
    \end{itemize}
 & Quantity \\\hline

 Promotion & Discounts for the products in a store & \begin{itemize}
        \vspace{-1.5em}
        \item Product (0,N)
        \item Store (0,N)
    \end{itemize}
 & Start-EndDates
 
 DiscountInfo \\\hline

 Supply & Associates a product with its supplier & \begin{itemize}
        \vspace{-1.5em}
        \item Product (1,1)
        \item Supplier (0,N)
    \end{itemize}
 & --- \\\hline

 Belongs & Associates a product with its category & \begin{itemize}
        \vspace{-1.5em}
        \item Product (1,1)
        \item Category (0,N)
    \end{itemize}
 & --- \\\hline

 Generated\_By & In which store was a receipt generated & \begin{itemize}
        \vspace{-1.5em}
        \item Receipt (1,1)
        \item Store (0,N) 
    \end{itemize}
 & --- \\\hline

 Works\_In & Associates a worker with the store they work in & \begin{itemize}
        \vspace{-1.5em}
        \item Employee (1,1)
        \item Store (1,N)
    \end{itemize}
 & --- \\\hline

 Supplies\_To & Associates a supplier with the stores they are currently supplying & \begin{itemize}
        \vspace{-1.5em}
        \item Supplier (0,N)
        \item Store (0,N)
    \end{itemize}
 & --- \\\hline

 Belongs\_To & Associates the customer with the receipt (if customer used the loyalty program while purchasing) & \begin{itemize}
        \vspace{-1.5em}
        \item Customer (0,N) 
        \item Receipt (0,1) 
    \end{itemize}
 & --- \\\hline

 Prefers & Which store does customer prefer & \begin{itemize}
        \vspace{-1.5em}
        \item Customer (0,1) 
        \item Store (0,N) 
    \end{itemize}
 & --- \\\hline

 Holds & Associates an employee with a position & \begin{itemize}
        \vspace{-1.5em}
        \item Employee (1,1)
        \item Position (0,N)
    \end{itemize}
 & --- \\\hline

\end{longtable}

\subsection{External Constraints}
\begin{itemize}
  \item A product can only belong to one category and can only be supplied by  one supplier.
  \item An employee can work at only one store and also can hold only one position at a time.
  \item A receipt can only be generated by one store and can be associated with at most one customer (if they are registered in the loyalty program and used the app during the purchase)
  
\end{itemize}



\subsection{Functional Requirements Satisfaction Check}
The database must store:
\begin{itemize}
  \item The details of the \textbf{Customer}, including the following attributes: name and surname, e-mail, password
  \item  The details of the \textbf{Product}, including the following attributes: Price, Product ID, UPC, VAT, name
  
  \item The details of the \textbf{Employee}, such as employee id, name, date, surname, hire date, salary, etc.
  
  \item The details of the \textbf{Receipt}, including the date, payment method, total amount and product name/names.
  
  \item The details of the \textbf{Supplier} including the SupplierID, Address, Phone, E-mail, contract details, etc. 
 \end{itemize} 
 
 \paragraph{}
The system must allow:
\begin{itemize}
\item \textbf{Customers} to login and signup.
  
\item \textbf{Stock Management}
\begin{itemize}
    \item Store managers to input stock quantities for products, to track reorder levels for products and to transfer stock from one store to another
    \item Stock clerks to track product stock levels within their store
\end{itemize}

\item \textbf{Purchase and Payment Management}
\begin{itemize}
    \item Customers to purchase products at stores
    \item Store managers to input purchase prices for products
    \item Customers to use their loyalty points while making a purchase
    \item Customers to view and download their receipts
    \item Store managers to track the payment methods used by customers
\end{itemize}
 
\item\textbf{Receipt and Transaction Management}
\begin{itemize}
    \item   Store managers to keep track and record unique receipt numbers for each purchase
    \item Customers to see transaction details, such as purchase quantity, date, and store location
\end{itemize}
   
\item\textbf{Product Information Management}
\begin{itemize}
    \item Stock clerks and Store Managers to input detailed product information, such as product ID, name, UPC, VAT, and price
    \item Inventory Control Specialists to manage product availability, including in-stock and out-of-stock statuses
\end{itemize}

\item\textbf{Store Information Management }
\begin{itemize}
    \item  Store managers to input detailed store information, such as store ID, name, location, schedule, and manager details
    \item Store managers to manage product supply chains, including supplier information and transfer schedules
\end{itemize}

\item\textbf{Category and Supplier Management }
\begin{itemize}
    \item  Store managers to input detailed category information, such as category name
    \item Store managers to input detailed supplier information, such as supplier ID, e-mail, address, phone and contract information
\end{itemize}

\item\textbf{Promotion and Discount Management}
\begin{itemize}
    \item  Store managers to input detailed promotion and discount information, such as discount type, percentage, or amount, start and end dates
    \item Store managers to associate specific promotions and discounts with individual products
\end{itemize}

\item\textbf{Reporting and Analysis}
\begin{itemize}
    \item Store managers to generate various types of reports, such as sales reports, inventory reports, and employee performance reports and also to analyze data from these reports to identify trends and make informed business decisions.
    \item Data Analysts to view the database and reports, such as sales reports, inventory reports and employee performance reports, in order to make statistics and to study the performance of supermarkets
\end{itemize}

\item\textbf{Security and User Management}
\begin{itemize}
    \item   Store managers to create and manage user accounts for employees, suppliers, and customers
    \item Store managers to assign appropriate user roles and permissions to each user account individual products
    \item Store managers to monitor and control user access to sensitive data and functions within the system
\end{itemize}
   
\end{itemize}